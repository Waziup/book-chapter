
\section{Use case applications}

The project will build a strong community able to address innovation in the African rural ecosystem through EU ICT technology.
Actors from various disciplines, including those having industrial backgrounds, those from application areas and target end-users will push towards the creation of an African-European innovation ecosystems based on the open sources for the IoT-Big data platform.
Starting from the rural domain, the project technology will allow the
African young application developer to produce new added value supporting multiple applications domains in the long run.
The induced ecosystem renews the business paradigm: new benefits are engendered from providing and sharing devices, services, and data across stakeholders and applications.

\subsection{Fish farming}

Two different test bed scenarios will be used to effectively integrate the real world fish farming situations/solutions with the digital world of WAZIUP.
The first test-bed is located in a small farm with 4 mini ponds of 90*90 metres and a large farm which operate several ponds of different sizes\footnote{http://www.kumahfarms.com/}.
These two farms produces both tilapia and catfish.
This use case validation will deploy sensor nodes responsible for sensing three parameters from the fish ponds: dissolved oxigen level, temperature and Ph.
The sensor nodes will communicate with the gateways which stores the data locally.
If the internet is available through Wifi of GPRS/UMTS, the data will be send to the Cloud instance of Waziup.

\subsection{Cattle breeding}

With the collaboration of the surrounding farmers we will put sensors around the cows’ neck in order to have measure related to location and speed.
We can also work with CIMEL(Centre d’Impulsion et de Modernisations de l’ELevage) which is a public structure for livestock impulsion and modernisation.
With this center we can make test and simulate a cattle rustling case.
To make this more interesting we imagine a smart food distribution system on CIMEL for a more optimal management of farmers’ ressources.

\subsection{Crop farming}

Weather surely plays a crucial role during the crop growth seasons.
Severe weather can impact crop growth by affecting different bio-physiological processes.
The impact on crop growth depends on the nature of extreme weather event, the crop type and time when it occurs.
Detrimental impact of extreme weather events may be alleviated by taking timely and appropriate preventive measures to cope with the nature and intensity of these weather events.
The availability of local weather predictions is a key factor in the decision process of the farmers.
For example, inputs that are added in the field (such as fertilizer of pesticides) can be lost if rain comes in the next few hours.
This use case presents how the WAZIUP platform can be used to prevent and alleviate the risks due to weather in farms.
To validate this use case, the WAZIUP project will build a low-cost weather station based on Arduino.
The data collected will be processed by the WAZIUP platform and will serve to adjust weather predictions on a local level.

For the deployment of the test bed, two pilots will be conducted in Ghana and Burkina Faso.
Both experimental farmer at the university and operational farm environment will be used as test-beds.
They will provide the possibility to interact directly with farmers, professionals and students who work on these farms.

\subsection{Supply chain management}

