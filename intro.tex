
\section{Introduction}

WAZIUP is a 3 years H2020 international cooperation action started on February 1st, 2016.
The project is driven by a consortium of 5 EU partners and of 7 partners from 4 sub-Saharan African countries.
Furthermore, it has support from multiple African stakeholders with the aim of defining new innovation space to advance the African Rural Economy.
It will do so by involving end-users communities in the loop, namely rural African communities of selected pilots, and by involving relevant public bodies in the project development.
WAZIUP will accelerate innovation in Africa by coupling with current EU innovation in the sector of IoT and Big Data: this EU technology will be specialized to generate African cost effective technologies with an eye to preparing the playground to the future technological waves by solving concrete current needs.
WAZIUP will deliver a communication and big data application platform and generate locally the know how by training by use case and examples.
The use of standards will help to create an interoperable platform, fully open source, oriented to radically new paradigms for innovative application/services delivery.


\subsection{ICT African context}

ICT, in Sub-Saharan Africa, in several cases has enabled convergence of productive sectors, serving as platform for more holistic development.
In fact, there are many examples of ICT developments in Africa that cut across traditional sectors: notable examples are the introduction of micro-health insurance and health-savings accounts through mobile devices; index-based crop insurance; crowd-sourcing to monitor and manage the delivery of public services, etc.
These innovative applications – for several reasons more disruptive in social terms than many counterparts in the EU - recognize and leverage commonalities between sectors, blur traditional lines, and open up a new field of opportunities.

The opportunity for ICT intervention in Africa is huge especially of IoT and big data: those technologies are promising a big wave of innovation for our daily life.
It is widely accepted that the Era of IoT can potentially connect billions of sensors, devices, equipment, systems, etc.
In turn, the challenge is about driving business outcomes, consumer benefits, and the creation of new value.
The new mantras for the IoT Era are becoming collection, convergence and exploitation of data.
The information collection involves data from sensors, devices, gateways, edge equipment and networks on to their respective siloed IoT platforms in order to increase process efficiency through automation while reducing downtime and improving people productivity.

WAZIUP targets the rural community in Sub-Saharan Africa because about 64\% of the population is living outside cities.
The region will be predominantly rural for at least another generation.
The pace of urbanization here is slower compared to other continents, and the rural population is expected to grow until 2045.
The majority of rural residents manage on less than few Euros per day.
Rural development is particularly imperative in sub-Saharan Africa, where half of the rural people are depend on the agriculture/micro and small farm business, other half faces rare formal employment and pervasive unemployment.
For rural development, technologies have to support several key application sectors like living quality, health, agriculture, climate changes, etc.
WAZIUP project consider how to best design and deploy the IoT-Big Data technology considering cost and energy challenges in the first place.

Beside the cost and power consumption, the robustness of hardware is a core requirement: hardware has to be robust enough so as to require lower maintenance and handle environmental and deployment threats as well.
WAZIUP will collect the grand challenge: reduce costs, reduce power consumption but at the same time increase the robustness of the hardware.

\subsection{Project challenges}

WAZIUP presents several challenges as listed below:
\begin{itemize}
  \item \emph{Innovative design of the IoT platform for the Rural Ecosystem.} 
    Low-cost, generic building blocks for maximum adaptation to end-applications in the context of the rural economy in developing countries.
  \item \emph{Network Management.}
    Facilitate IoT communication and network deployment.
    Lower cost solutions compared to state of the art technology: privilege price and single hop dedicated communication networks, energy autonomous, with low maintenance costs and long lasting operations.
  \item \emph{Long distance.}
    Dynamic management of long range connectivity (e.g., cope with network \& service fluctuations), provide devices identification, abstraction/virtualization of devices, communication and network resources optimization.
  \item \emph{Big-data.}
    Exploit the potential of big-data applications in the specific rural context.
\end{itemize}

From a technical standpoint, WAZIUP introduces innovation by constructing on the following pillars of IoT/Big Data technology, specifically tailored for the rural ecosystem:
\begin{itemize}
  \item \emph{Privacy and security:} through attention to all related privacy and security aspects with specifics addressing the involved communities (farmers, developers);
  \item \emph{Personalized and user friendliness:} models will receive requirements from users’ needs and will ensure compliance with all most common usability standards (e.g., Web Accessibility Initiative - WAI or ISO/TR 16982:2002);
  \item \emph{An Open interoperable platform:} through open standard and protocols from the Geospatial Consortium (OGC), W3C, IEEE from the European SDOs (CEN, CENELEC and ETSI, etc.) for all its key technology
  \item \emph{Continued Openness:} through the release of open specification and open software components and/or algorithms;
  \item \emph{Low-cost and low-energy consumption:} through the design of innovation hardware (sensors/actuators), and of IoT communication \& network infrastructure.
\end{itemize}


