
\section{Conclusion}
\label{sec:conclu}

With ICT technologies, Africa can dramatically improve its agricultural productivity by enabling the rapid and cost-effective deployment of advanced and real time monitoring.
The immediate effect is to improve coordination and logistics, by reducing time and investment horizons for Research and Development and new product development, and by allowing for the enhanced analysis of historical and ongoing data. 
With respect to the water sector, such systems can also dramatically improve water use efficiency, allow for the growth of water provider SMEs by providing practical and cost effective new payment, monitoring and management systems.
This technology can also offer a new cost-effective alternative for integrated watershed management by networking real-time water quality and flow data. 
Furthermore, given the fundamental roles which agriculture and water play in the African economic and social development and more generally onto environmental sustainability, WAZIUP can both directly and indirectly bring a much wider range of benefits related to food security, gender equality, poverty reduction and resource use efficiency.
